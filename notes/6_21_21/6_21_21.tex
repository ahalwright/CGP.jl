\documentclass{article}
\usepackage[utf8]{inputenc}

\title{Degeneracy by increasing gates}
\author{Alden Wright }
\date{June 21, 2021}

\usepackage{natbib}
\usepackage{graphicx}
\setlength{\parindent}{0mm}
\setlength{\parskip}{1ex}
\textwidth=15.7cm
\textheight=22.9cm
\voffset=-2.54cm
\hoffset=-2cm
\begin{document}

\maketitle

\section{Introduction}
Ideas on degeneracy.

As shown in 6\_20\_21.txt, degeneracy and robustness increase with the number of gates.
Better summarized these results in a spreadsheet in
evotech/complexity/data/6\_21\_21/degeneracy\_by\_increasing\_gates.xlsx

Hypothesis:  increased degeneracy increases "resiliancy".

Hypothesis:  Start with a circuit that maps to a given phenotype.  Circuit failure
is simulated by mutating the circuit.  Measure the number of evolutionary steps
for the circuit to recover the phenotype.  Implementation questions:
Run with a large step limit, or do multiple runs with smaller step limits.
Note that if robustness is large, then many mutations won't change the phenotype.

Hypothesis:  Start with a circuit that maps to a given phenotype.  Mutate the phenotype,
and measure how long it takes for the circuit to evolve to the mutated phenotype.
Note there is much evolutionary computation research on varying goals.  But for now,
look at straighforward techniques.  Again, there is the question of running with a
large step limit, or doing multiple runs with a smaller step limit.

The paper "Degeneracy: A design principle for achieving robustness and evolvability"
by Whitacre, Bender, 2010 
\cite{RefWorks:doc:5e800362e4b03e5926388ac6} 
addresses many of these questions, and my document
complexity/cce/'Degeneracy preliminary proposal 8\_21\_20.docx' gives my earlier thoughts
(not much original in this document).                         

\section{Conclusion}


\bibliographystyle{plain}
\bibliography{refworks6_21_21.bib}
\end{document}
